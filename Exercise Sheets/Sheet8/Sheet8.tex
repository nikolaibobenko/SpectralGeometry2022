\documentclass[a4paper,11pt]{article}
\usepackage[T1]{fontenc}
% \usepackage[french]{babel}
% \usepackage[latin1]{inputenc}
%\usepackage{umlaut,amssymb,amsmath,amscd,a4,amsfonts}
\usepackage{amssymb,amsmath,amscd,a4,amsfonts,amsthm}
\usepackage{MnSymbol, exercise}
%(a4 = 210 X 297 mm)
\hoffset -1in \voffset -1in \oddsidemargin 20mm \evensidemargin
\oddsidemargin \textwidth 170mm \topmargin 5mm \textheight 247mm

\newtheorem{theorem}{Theorem}
\newtheorem{lemma}{Lemma}

\theoremstyle{definition}
\newtheorem{exercise}{Exercise}


\begin{document}

\pagestyle{headings}
\noindent UNIVERSITE DE GENEVE \hfill Section de Mathématiques\\
\noindent Facult\'e des sciences \hfill \\[-3mm]
\hrule

\large

\begin{center}
\textbf{Spectral Geometry - SS 2022 \\ Sheet 8 - Discussed on 04.05.2022}
\end{center}
\hrule
\text{}\\[1cm]

Obligatory exercises are marked with a $(*)$. We ask you to solve these and be ready to present your solutions during the exercise session. Let me know one day in advance which ones you were able to solve and I will randomly assign some of you to present your solutions. You will get points for the exercises you announced as solved with possible deductions if your presentation is lacking.

\begin{exercise}
	Let $\Delta u + u = 0$ in $\mathbb{R}^2$. Assume that the nodal curves set $Z_u$ is a collection of smooth curves. Show that
	\begin{enumerate}
		\item $|Z_u|$ is infinite.
		\item $|Z_u \cap B_r| \geq c r^2$ for $r$ large enough.
	\end{enumerate}
\end{exercise}

\begin{exercise}
	We want to show that for the following tpes of domains there exists a constant $C = C(\Gamma)$ such that $\int_\Gamma |\varphi_\lambda|^2 \leq C \int_{\Pi^2} |\varphi_\lambda|^2$.
	\begin{enumerate}
		\item $\Gamma $ is a vertical segment
		\item $\Gamma$ is any closed geodesic.
	\end{enumerate}
\end{exercise}

\begin{exercise}
	Use the theorem of Bourgain and Rudnick to show that only finitely many eigenfunctions on $\Pi^2$ can vanish on a curve which is not a closed geodesic.
\end{exercise}

\begin{exercise}
	Let $u \in C^\infty(\mathbb{R}^2)$ with $u = \text{Re}(z^k) + O(|z|^{k+1})$ and $\nabla u = \nabla \text{Re}(z^k) + O(|z|^{k})$. Show that then locally around $0$ the zero set of $u$ looks like the zero set of $\text{Re}(z^k)$.
\end{exercise}

\end{document}
